

\begin{table*}[tbh!]
\normalsize
\renewcommand\arraystretch{1.2}
\centering
    \caption[在MSR-VTT和TRECVID上联合评测目标函数不同的模型和文本编码器融合方式不同的模型]{\textbf{在MSR-VTT和TRECVID上联合评测目标函数不同的模型和文本编码器融合方式不同的模型}。评价指标除了中位排序数Med r均以百分数的形式显示(\%),其中最好的结果用\best{粗体}标出。本文提出的W2VV++模型对基础模型W2VV的目标函数的改进起到了很大的作用,并且也更有效地使用了GRU编码器,而使用多空间多目标函数融合文本编码器的SEA模型的性能最好。}
\label{tab:loss_fusion}
\scalebox{0.79}{
\begin{tabular}{@{}|l | r | r | r | r | l|| r | r | r | r | l|@{}}
\hline
\multirow{2}{*}{\textbf{模型}} & \multicolumn{5}{c||}{\textbf{MSR-VTT} (test-full)} & \multicolumn{5}{c|}{\textbf{TRECVID} (指标: infAP)}  \\
\cline{2-11}
 & \textit{R@1} & \textit{R@5} & \textit{R@10} & \textit{Med r} & \textit{mAP} & \textit{TV16} & \textit{TV17} & \textit{TV18} & \textit{TV19} & \textit{AVG}  \\

\hline
W2VV & 1.1 & 4.7 & 8.0 & 240 & \textcolor{white}{0}3.7 & 1.3 & 0.8 & 0.4 & 0.2 & \textcolor{white}{0}0.7 \\

\cline{1-11}
W2VV$_{ITRL}$ & 9.7 & 27.2 & 37.3 & 22 & 18.7 & 13.8 & 18.8 & 10.5 & 11.4 & 13.6 \\

\cline{1-11}
W2VV++ &  11.1 & 29.6 & 40.5 & 18 & 20.6  & \best{16.2} & 22.3 & 10.1 & 13.9 & 15.6 \\

\cline{1-11}
Transformed W2VV++ & 10.5 & 28.2 & 39.0 & 20 & 19.5 & 13.9 & 20.2 & 10.2 & 13.5 & 14.5 \\

\cline{1-11}
Model averaging & 12.0 & 31.8 & 42.9 & 16 & 22.0  & 14.9 & 21.9 & 11.6 & 15.4 & 16.0 \\

\cline{1-11}
SEA single loss & 11.8 & 31.0 & 42.0 & 17 & 21.5  & 14.7 & 21.8 & 11.2 & 14.7 & 15.6 \\

\cline{1-11}
SEA combined loss & \best{12.2} & \best{31.9} & \best{43.1} & \best{15} & \best{22.1}  & 15.0 & \best{23.4} & \best{12.2} & \best{16.6} & \best{16.8} \\


\hline 

\end{tabular}
}
\end{table*}


