\chapter{实验}
\section{算法实现}
本研究使用PyTorch~\cite{}的深度学习框架实现W2VV++算法模型和TEE算法模型,使用RMSProp~\cite{}优化器进行模型的训练,优化器的学习率
根据实验经验设为0.0001,而其他参数使用默认值。为了防止训练时出现梯度爆炸,本研究把训练时的梯度降低l2范数倍。学习率在每次训练结束
后降为原来的0.99倍,如果模型在验证集上的性能连续三次没有提高则学习率降为原来的0.5倍。如果在验证集的性能连续10次没有提高,则模型训练
停止。本研究选择在验证集上性能最好的模型。

每次模型训练批次的大小为128对相关的句子-视频对,在每个批次给定的一个句子,本研究简单把该批次里的其余剩余视频当作该句子的不相关视频,
即该句子与这些不相关视频构成负样本,而本研究使用的最难负例来计算损失的策略,即选择这些负例中的最不相关的句子视频对来最后计算损失,
具体是余弦相似度最大的句子视频对作为最难负例。而公式\ref{eq:loss}损失函数的超参数$\alpha$设为0.2。为了避免出现过拟合,本研究在转换网络的全连接层使用概率为0.2的随机失活技术。

\section{实验数据}
为了验证本研究的算法的有效性,本研究设置了两组实验,一组是在TRECVID AVS任务上的,因为本研究的本质也是文本检索视频的任务,因此在
具有权威的视频检索数据集msrvtt10k上的实验验证本研究模型的有效性。

\subsection{TRECVID AVS实验数据}
本研究使用的训练、验证与测试集的一些基本统计如表格\ref{tab:dataset1}所示。

\textbf{训练集}:本研究使用MSR-VTT~\cite{}和TGIF~\cite{}两个数据结合作为训练集。MSR-VTT数据包含1万个网络视频片段和20万个描述视频片段内容的句子,即每个视频片段有20个描述句子。
本研究对每段视频进行均匀采样,共生成305,462帧图像。而TGIF数据包含超过10万张动图和12万个描述动图内容的句子。本研究同样对每个动图进行均匀采样,生成1,045,268帧图像。

\textbf{验证集}:本研究使用TRECVID 2016 Video-to-Text任务~\cite{}的训练集作为验证集,命名为TV16-VTT-train,用于模型训练阶段的
最优模型选择。这个集合共包含200个视频,每个视频含有2个描述视频内容的句子。对于每个视频,本文选择第一个句子作为该视频的文本查询,
而剩余的199个视频则是该查询的不相关视频。相应地,本研究使用平均倒数排名(mean reciprocal rank, MRR)作为评价模型在该验证集性能的指标。
\begin{table} [tbh!]
    \caption[AVS数据集]{这是AVS数据集的基本统计描述}
    \label{tab:avs-dataset}
    \centering
    \scalebox{0.96}{
        \begin{tabular}{@{}l r r r r r@{}}
            \toprule
            \textbf{数据集名称} & \textbf{镜头数} & \textbf{帧数} & \textbf{句子数} & \textbf{bow词典大小} & \textbf{GRU词典大小} \\
            \hline
            \emph{训练集:} & & & & & \\
            MSR-VTT & 10,000 & 305,462 & 200,000 & \multirow{2}{*}{11,147} & \multirow{2}{*}{11,282} \\
            TGIF & 100,855 & 1,045,268 & 124,534 & & \\
            \hline
            \emph{验证集:} & & & & & \\ 
            TV16-VTT-train & 200 & 5,941 & 200 & - & - \\
            \hline
            \emph{测试集:} & & & & & \\
            IACC.3 & 335,944 & 3,845,221 & - & - & - \\
            V3C1 & 1,082,649 & 7,839,450 & - & - & - \\
            \bottomrule
        \end{tabular}
    }
\end{table}


\textbf{视频测试集}:本研究使用TRECVID AVS任务的官方测试集IACC.3(2016-2018)~\cite{}和V3C1(2019)~\cite{}。IACC.3数据集包含4,953个网络视频(600小时),视频的时长从6.5分钟到9.5分钟,平均时长为7.8分钟。
官方已经对视频做了自动镜头边界检测,共生成335,944个视频片段。本研究同样对每个视频片段进行均匀采样,共生成3,845,221帧图像。

\textbf{查询测试集}:TRECVID AVS官方每年定义30个查询来进行测试,每个查询是自然语言的句子形式,具有不同的长度和不同的语义难度。
例如“Find shots of palm trees”,“Find shots of a man with beard and wearing white robe speaking and gesturing to camera”和
“Find shots of a truck standing still while a person is walking beside or in front of it”。所有的查询均以“Find shots of”的短语开头,因此在测试时可以很容易地去掉。

\textbf{性能指标}:本研究使用TRECVID AVS任务官方的评测指标,推断平均准确率(inferred average precision, infAP)~\cite{}作为该节实验的模型评价指标,而模型的总体性能是所有测试查询的infAP分数的平均值,该值越高,模型的视频检索效果越好。







\section{实验结果}
\subsection{插图表格}
\begin{figure}[htbp]
\centering\includegraphics[width=5cm,height=1.32cm]{figures/logo3.pdf}
\caption[中英校名]{中英校名}
\end{figure}
\begin{table}[htbp]
\noindent\begin{minipage}{0.5\textwidth}
\centering
\caption{并排子表格}
\label{tab:parallel1}
\begin{tabular}{p{2cm}p{2cm}}
\toprule[1.5pt]
姓名 & 性别 \\\midrule[1pt]
李狗蛋 & 女 \\\bottomrule[1.5pt]
\end{tabular}
\end{minipage}
\begin{minipage}{0.5\textwidth}
\centering
\caption{并排子表格}
\label{tab:parallel2}
\begin{tabular}{p{2cm}p{2cm}}
\toprule[1.5pt]
姓名 & 性别 \\\midrule[1pt]
张狗蛋 & 女 \\\bottomrule[1.5pt]
\end{tabular}
\end{minipage}
\end{table}
\begin{table}[htbp]
\centering
\caption{并排子表格}
\label{tab:subtable}
\subtable[第一个子表格]
{
\begin{tabular}{p{2cm}p{2cm}}
\toprule[1.5pt]
姓名 & 性别 \\\midrule[1pt]
田狗蛋 & 男 \\\bottomrule[1.5pt]
\end{tabular}
}
\hskip2cm
\subtable[第二个子表格]
{
\begin{tabular}{p{2cm}p{2cm}}
\toprule[1.5pt]
姓名 & 性别 \\\midrule[1pt]
李狗蛋 & 女 \\\bottomrule[1.5pt]
\end{tabular}
}
\end{table}

\subsection{数学环境}
下面是几个数学公式的例子:\par
\begin{equation}
\begin{aligned}
P\{S_n \leq t\} &= \int_{-\infty}^{+\infty}f_{S_n}dt \notag \\
                       &= \int_0^t\frac{\lambda(\lambda u)^{n-1}}{(n-1)!}e^{-\lambda u}du \\
                       &\xlongequal{令 \lambda u=x} \frac{1}{(n-1)!}\int_0^{\lambda t}x^{n-1}e^{-x}dx\\
                       &=\frac{-1}{(n-1)!}(e^{-x}x^{n-1}{\Big|}_0^{\lambda t}-\int_0^{\lambda t}e^{-x}dx^{n-1})\\
                       &=\frac{-1}{(n-1)!}e^{-x}x^{n-1}{\Big|}_0^{\lambda t}+\frac{1}{(n-2)!}\int_0^{\lambda t}e^{-x}x^{n-2}dx
\end{aligned}
\end{equation}\par
再来几个:
\begin{equation}
\begin{aligned}
\lambda &=\left (1+\frac{\left(\frac{\bar{X}-\bar{Y}}{\sqrt{((\frac{1}{n}+\frac{1}{m})\sigma^2)}}\right)^2}{\left(\sqrt{\frac{\sum\limits_{i=1}^n(X_i-\bar{X})^2+\sum\limits_{i=1}^m(Y_i-\bar{Y})^2}{(m+n)\sigma^2}}\right)^2(m+n-2)}\right)^{\frac{n+m}{2}} \\ \notag
            &=\left(1+\frac{T^2}{n+m-2}\right)^{\frac{n+m}{2}}\\
 其中\quad T^2 &=\left(\frac{\frac{\bar{X}-\bar{Y}}{\sqrt{((\frac{1}{n}+\frac{1}{m})\sigma^2)}}}{{\sqrt{\frac{\sum\limits_{i=1}^n(X_i-\bar{X})^2+\sum\limits_{i=1}^m(Y_i-\bar{Y})^2}{(m+n)\sigma^2}}}}\right)^2
\end{aligned}
\end{equation}
