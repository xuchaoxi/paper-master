
\begin{table} [tbp!]
    %\renewcommand{\arraystretch}{0.95}
    \caption[AVS系统回顾]{该表格回顾了2016年至2019 TRECVID的排名前三名的AVS系统}
    \label{tab:method-diffs}
    \centering
    \scalebox{0.75}{
        \begin{tabular}{| l | l | l | l | l |}
            %\toprule
            \hline
            \textbf{排名} & \textbf{系统} & \textbf{文本表示} & \textbf{视频表示} & \textbf{公共空间} \\
            \hline
            \multicolumn{5}{|l|}{2016:} \\ \hline
            1 & Le \etal\cite{le2016nii} & \specialcell{词袋向量(Bag-of-\\words~\cite{sivic2003video})} & \specialcell{通过预训练的卷积神经网络\\(VGG-16~\cite{simonyan2014very})提取的概念特\\征向量} & 文本特征空间 \\ \hline
            2 & Markatopoulou \etal\cite{foteini2016iti} & \specialcell{基于规则的概念选择得\\到的概念特征向量~\cite{markatopoulou2017query}} & \specialcell{通过预训练的卷积神经网络\\(AlexNet~\cite{krizhevsky2012imagenet},GoogLeNet~\cite{szegedy2015going}, \\ResNet~\cite{he2016deep}, VGGNet~\cite{sivic2003video})\\提取的概念特征向量} & \specialcell{1345 维的概念\\特征空间} \\ \hline
            3 & Liang \etal\cite{liang2016inf} & word2vec~\cite{mikolov2013distributed} & \specialcell{通过预训练的卷积神经网络\\(VGG-19~\cite{simonyan2014very}, C3D~\cite{tran2015learning})\\提取的视觉特征} & 视觉特征空间 \\ \hline
            \multicolumn{5}{|l|}{2017:} \\ \hline
            1 & Snoek \etal\cite{snoek2017university} & \specialcell{词袋向量(Bag-of-\\words)~\cite{sivic2003video}} & \specialcell{由VideoStory~\cite{habibian2014videostory}模型产生\\的词袋向量(Bag-of-words)} & \specialcell{文本特征空间} \\ \hline
            2 & Ueki \etal\cite{ueki2017waseda} & \specialcell{基于规则的概念选择得\\到的概念特征向量} & \specialcell{通过预训练的卷积神经网络\\(AlexNet~\cite{krizhevsky2012imagenet}, GoogLeNet~\cite{szegedy2015going})\\提取的概念特征向量} & \specialcell{5万维的概念\\特征空间} \\ \hline
            3 & Nguyen \etal\cite{nguyen2017vireo} & \specialcell{基于规则的概念选择得\\到的概念特征向量} & \specialcell{通过预训练的卷积神经网络\\(ResNet-50)提取的概念特征\\向量} & \specialcell{2774维的概念\\特征空间} \\ \hline
            \multicolumn{5}{|l|}{2018:} \\ \hline
            1 & Li \etal\cite{li2018renmin} & \specialcell{由W2VV++~\cite{li2019w2vv++}模型产\\生的稠密向量} &  \specialcell{通过预训练的卷积神经网络\\(ResNeXt-101~\cite{xie2017aggregated}, ResNet-\\152)提取的视觉特征} & \specialcell{视觉特征空间/\\学习的隐空间} \\ \hline
            2 & Huang \etal\cite{huang2018informedia} & \specialcell{+基于规则的概念选择\\得到的概念特征向量\\+通过注意力网络得到\\的稠密向量} & \specialcell{+通过预训练的卷积神经网\\络提取的概念特征向量\\+通过预训练的卷积神经网\\络提取的视觉特征} & \specialcell{+概念特征空间\\+学习的隐空间} \\ \hline
            3 & Bastan \etal\cite{bastan2018ntu} & \specialcell{由VSE++~\cite{faghri2017vse++}模型得到\\的稠密向量} & \specialcell{通过预训练的卷积神经网络\\(ResNet-152)提取的视觉特征} & \specialcell{学习的隐空间} \\ \hline
            \multicolumn{5}{|l|}{2019:} \\ \hline
            1 & Wu \etal\cite{wu2019hybrid} & \specialcell{混合文本编码策略} & \specialcell{混合视觉特征编码策略} & \specialcell{学习的隐空间} \\ \hline
            2 & Li \etal\cite{li2019renmin} & \specialcell{由W2VV++~\cite{li2019w2vv++}模型产\\生的稠密向量} & \specialcell{通过预训练的卷积神经网络\\(ResNeXt-101, ResNet-152)\\提取的视觉特征} & \specialcell{视觉特征空间/\\学习的隐空间} \\ \hline
            3 & Ueki \etal\cite{ueki2019waseda} & \specialcell{由VSE++~\cite{faghri2017vse++}模型得到\\的稠密向量} & \specialcell{通过预训练的卷积神经网络\\(ResNet-50, ResNet-101,\\ ResNet-152)提取的视觉特征} & \specialcell{学习的隐空间} \\ \hline
            %\bottomrule
        \end{tabular}
    }
\end{table}
