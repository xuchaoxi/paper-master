\chapter{绪论}
%\section{安装\LaTeX{} }
%\subsection{Mac OS X}
%\begin{figure}[htbp]
%\centering\includegraphics[width=5cm,height=1.32cm]{figures/Logo_2.pdf}
%\caption[示意图]{用LaTeX画图}
%\end{figure}

\section{研究背景与意义}
\subsection{即席视频检索检索}
随着大数据时代的进一步发展,数字化与智能化进程不断加速,新时代的大数据呈现出新的特点与挑战。

即席视频检索检索(Ad-hoc Video Search, AVS)是指用户根据自己的需求以句子的形式来查询未经标注的视频,例如“一个有胡子的男人对着麦克风讲话或唱歌”。这在多媒体检索领域里是一个非常重要但又十分有挑战的问题,因为视频和文本这两个不同的模态存在语义鸿沟。

随着网络时代的发展,越来越多的视频应用出现,如 Youtube、爱奇艺、抖音等,用户可以随意上传短视频甚至电影,视频数据呈爆炸性的增长,其中一个重要的需求就是从海量的视频数据里精确地检索到用户需要的视频。通常情况下,用户没有要查找的视频样例,只能通过文本的形式来表达其需求。因此,研究以文本的形式检索视频,提高视频检索的准确率具有深远的理论意义和实践意义。


\subsection{文本表示}
随着大数据时代的进一步发展,数字化与智能化进程不断加速,新时代的
大数据呈现出新的特点与挑战
\section{本文研究内容}
随着大数据时代的进一步发展,数字化与智能化进程不断加速,新时代的
大数据呈现出新的特点与挑战
\section{论文结构安排}
随着大数据时代的进一步发展,数字化与智能化进程不断加速,新时代的
大数据呈现出新的特点与挑战

