% XeLaTeX can use any Mac OS X font. See the setromanfont command below.
% Input to XeLaTeX is full Unicode, so Unicode characters can be typed directly into the source.

% The next lines tell TeXShop to typeset with xelatex, and to open and save the source with Unicode encoding.

%!TEX TS-program = xelatex
%!TEX encoding = UTF-8 Unicode

\documentclass[a4paper,master]{ructhesis}

%自添宏包
\usepackage{skak}%国际象棋
\usepackage{subfigure}%子图http://www.ctex.org/documents/latex/graphics/node111.html
\usepackage{chemfig}%化学式
\usetikzlibrary{trees}

\usepackage{bm}
\usepackage{longtable}
\usepackage{supertabular}
\usepackage{enumerate}
%解决目录红框
\hypersetup{
colorlinks=true,
linkcolor=black
}
\usepackage{eucal}
\newcommand{\best}[1]{\textcolor{blue}{\textbf{#1}}}
\newcommand{\tea}{\textit{SEA~}}

\newcommand{\ie}{\emph{i.e.,~}}
\newcommand{\etal}{\emph{et al.~}}
\newcommand{\eg}{\emph{e.g.,~}}
\newcommand{\wrt}{\emph{w.r.t\onedot~}}

\newcommand{\fig}[1]{Figure~\ref{fig:#1}}
\newcommand{\sect}[1]{Section~\ref{sec:#1}}
\newcommand{\Eq}[1]{Eq.~(\ref{eq:#1})}
\newcommand{\eq}[1]{\Eq{#1}}
\newcommand{\tab}[1]{Table~\ref{tab:#1}}
\newcommand{\alg}[1]{Algorithm~\ref{alg:#1}}

\newcommand{\specialcell}[2][l]{%
      \begin{tabular}[#1]{@{}l@{}}#2\end{tabular}}


%将封面信息补全,相关专业名称过长的请在文字前添加命令\ziju{-0.15}
%文头
%\sign{中国人民大学本科毕业论文}
\sign{硕士学位论文}
%\sign{博士学位论文}

%以下信息本科研究生都需要补全
\title{基于深度学习的即席视频检索}%论文题名
\author{徐朝喜}%作者
\school{信息学院}%学院
\field{\ 计算机应用技术}%专业
\studentid{2017104074}%学号
\advisor{李锡荣}%指导老师
\date{2020年3月17日}
%以下本科填写
\grade{}%年级
\score{4.0}%成绩
\thesiscode{论文编码:RUC-BK-专业代码}%论文编码
%\subtitle{------20天打虎从入门到精通}%论文副题名
%以下研究生填写
\etitle{Ad-hoc Video Search Based on Deep Learning}%英文题目
\keywords{\TeX{}}%论文主题词
%摘要关键词
\keywordzh{中文摘要关键词}%中文摘要关键词
\keyworden{English\qquad template}%英文摘要关键词

%
\begin{document}

%扉页
\maketitle

%独创性声明
\originality
%授权书在这插入
\authorization{figures/shouquan.png}
%中文摘要
% the abstract
\begin{abstractzh}
    即席视频检索在多媒体检索领域里是一个十分重要但是很有挑战的一个问题。不同于之前基于概念建模的方法,我提出了一个完全基于深度学习的方法来对查询语句的表示进行建模。该方法不需要显式的概念建模、匹配和选择。我的方法是以W2VV++为基础了,W2VV++方法是一个用于图像文本匹配模型Word2VisualVec(W2VV)模型的改进版。W2VV++是对W2VV的句子编码策略改进和使用improved triplet ranking loss的损失函数替代原来的均方差损失函数。通过这些简单但有效的改进,W2VV++的效果有了很大的提高。我们通过参加TRECVID 2018 AVS任务并且通过在TRECVID 2016和2017数据上的实验,我们最好的模型以总体infAP为0.157的性能,是最好的模。
\end{abstractzh}

%英文摘要
% the abstract
\begin{abstracten}
    Ad-hoc video search (AVS) is an important yet challenging problem in multimedia retrieval. Different from previous concept-based methods, we propose a fully deep learning method for query representation learning. The proposed method requires no explicit concept modeling, matching and selection. The backbone of our method is the proposed W2VV++ model, a super version of Word2VisualVec(W2VV) previously developed for visual-to-text matching. W2VV++ is obtained by tweaking W2VV with a better sentence encoding strategy and an improved triplet ranking loss. With these simple yet important changes, W2VV++ brings in a substantial improvement. As our participation in the TRECVID 2018 AVS task and retrospective experiments on the TRECVID 2016 and 2017 data show, our best single model, with an overall inferred average precision (infAP) of 0.157, outperforms the state-of-the-art. The performance can be further boosted by model ensemble using late average fusion, reaching a higher infAP of 0.163. With W2VV++, we establish a new baseline for ad-hoc video search.

\end{abstracten}



\frontmatter

%正文目录
\tableofcontents
%插图目录
\listoffigures
%表格目录
\listoftables


\mainmatter\clearpage
\pagestyle{fancy}

%正文章节
\chapter{绪论}
%\section{安装\LaTeX{} }
%\subsection{Mac OS X}
%\begin{figure}[htbp]
%\centering\includegraphics[width=5cm,height=1.32cm]{figures/Logo_2.pdf}
%\caption[示意图]{用LaTeX画图}
%\end{figure}

\section{研究背景与意义}
随着移动互联网的快速发展,人们可以随时随地通过智能设备上的应用创建和分享各种不同模态的多媒体数据,如文本、图像、视频等。其中视频
数据具有更丰富的信息,受到了更多用户的青睐,据《第44次中国互联网络发展状况统计报告》~\footnote{http://www.cac.gov.cn/2019-08/30/c\_1124938750.htm}显示,截至2019年6月,我国的网络视频用户规模
达7.59亿,占网民整体的88.8\%,其中短视频用户规模为6.48亿,占网民整体的75.8\%。各种短视频的智能设备应用也是呈井喷式的发展,如抖音、
快手等应用,用户可以通过这些应用创建和分享一分钟以内的短视频,据《2019年抖音数据报告》~\footnote{https://weibo.com/ttarticle/p/show?id=2309404457716281114897}显示,截至2020年1月5日,抖音日活跃用户数已经超过4亿,而根据快手发布的《2019快手内容报告》~\footnote{http://www.chinanews.com/business/2020/02-22/9100846.shtml}
显示,2019年有2.5亿用户通过快手平台发布作品,平台内有近200亿条的海量视频数据。在如此庞大的并且还在快速增长的视频数据里,
如何高效和精确地检索出用户需要或者感兴趣的视频是一个具有很大挑战并且具有应用价值的问题,是多媒体检索领域内的热点问题~\cite{hong2017,geetha2008a,hu2011a,peng2018an}。

即席视频检索检索(Ad-hoc Video Search, AVS)是指用户根据自己的需求以句子的形式来查询未经标注的视频,例如“一个有胡子的男人对着麦克风讲话或唱歌”~\cite{awad2016trecvid}。这和经典的基于内容的视频检索~\cite{yu2015content}不同,基于内容的视频检索是以一个图像或者视频片段
作为查询,从候选的视频数据中检索出与该查询内容相同或者相似的视频。这种视频检索方式在实际应用中有很大的局限性,因为用户通常不具有
能够表达其查询需求的图片或者视频,而以句子的形式表达用户需求则是一种更加直接方便的方式,例如用户希望检索出“一个男人在唱歌”这样场景的视频,而其并没有这样的样例视频,则其不能通过这种基于视频内容的视频检索技术进行检索,因此用户会更加倾向于使用句子的形式作为查询来检索想要的视频。即席视频检索也不同于长期研究的基于概念的视频检索技术~\cite{snoekcees2009concept},这种检索技术的和核心是从视频中
检测出某些特定的概念,例如视频中出现的人或者物体,然后对查询句子提取关键词,利用查询关键词与检测出的概念进行匹配,这种检索技术需要
先定义一个概念集,然后检测视频中出现的概念,因此检索效果受限于概念的选择并且这种技术对查询句子的语义也没有进行建模分析。而本文研究
的即席视频检索是解决用户以一种模态的多媒体对象(文本)作为查询输入,检索出与查询语义上相同或相近的另一种模态的多媒体对象(视频),
这本质是也是一直被广泛关注的跨模态检索技术~\cite{rasiwasia2010a,feng2014cross,pereira2014on,suris2018cross,mithun2018learning}。
即席视频检索的核心是计算查询文本与候选视频在语义上的相关度,然后根据该相关度对候选视频进行排序,从而从候选视频中选出用户期望的视频,图~\ref{fig:fig_1}展示了即席视频检索的大致流程。由于查询文本与视频是两种不同的模态,它们在底层表示上是异构的,即文本是由一序列的单词排列组成,而视频是由一序列图像组成并且伴随着音频,这也通常被称为“异构鸿沟”,因此这两种数据是不能直接进行比较或者计算相关度的。这两种
媒体数据的异构性和不可比性,使得即席视频检索是一个非常具有挑战性的研究任务。得益于美国国家标准与技术研究院(National Institute of Standards and Technology, NIST)举行的视频检索国际权威评测TRECVID AVS,提供的大量的测试数据集合、统一的评价标准,吸引了来自包括卡内基梅隆大学、弗吉尼亚大学、香港中文大学、早稻田大学等全球各地的优秀学者参与评测,推动了即席视频检索技术的发展~\cite{awad2016trecvid,awad2017trecvid,awad2018trecvid,awad2019trecvid}。如表格\ref{tab:method-diffs}所示,最近四年的研究方法都是基于公共空间学习的方法,这种方法为文本和视频学习一个公共空间,并将这两种模态的数据投影到公共空间进行表示,使得这两种数据的相关度可以在这个公共空间通过直接计算距离进行衡量,
这种方法有三个关键的模块,即文本表示、视频表示和公共空间,这三个模块的好坏与检索算法的效果有直接的关系,本文将关注对文本表示和公共空间这两个模块的改进,而视频表示则采用简单但有效的通过预训练的卷积神经网络(ResNeXt-101, ResNet-152)提取的视觉特征~\cite{li2018renmin,li2019renmin}。

\begin{figure*}[tbh!]
    \centering
    \includegraphics[width=\linewidth]{figures/1}
    \caption[即席视频检索流程]{\textbf{即席视频检索流程}。即席视频检索技术首先计算查询与候选视频的相关度,然后根据相关度对候选视频进行排序,从而得到最终的视频检索结果。本文使用余弦相似度来度量查询与候选视频的相关度,范围从-1到1,余弦相似度越大,则表明查询与候选视频的相关度越高。}
    \label{fig:fig_1}
\end{figure*}

%随着深度学习的提出~\cite{hinton2006a},
在ImageNet图像识别大赛中~\cite{deng2009imagenet},Hinton等人在文献~\cite{krizhevsky2012imagenet}提出的深度卷积网络Alexnet把图像识别错误率从原来的25.7\%提升到15.3\%,极大地推动了深度学习模型的发展,而后一系列的网络层数更深的深度卷积网络VGGNet~\cite{simonyan2014very},
GoogLeNet~\cite{szegedy2015going},ResNet~\cite{he2016deep}将图像识别错误率逐步提升到3.57\%,超越了人类的图像识别错误率。
深度学习模型在图像领域的巨大成功,深度学习的方法也被拓展到视频识别领域。考虑到视频比图像而言不仅具有空间信息,也具有时序信息,Simonyan等人在文献~\cite{simonyan2014two}提出双流的深度卷积网络来处理视频的动作识别任务,同时以视频的帧图像和光流信息作为深度卷积网络的输入,证明了深度学习模型在视频动作识别领域比传统的基于人工的特征,如方向梯度直方图(Histogram of Oriented Gradient,HOG)和光流直方图(Histogram of Optical Flow,HOF)的效果相当。因为视频有时序性的特定,后来Du等人在文献~\cite{tran2015learning}上基于图像上的二维卷积核的基础推广到视频上的三维的卷积核,提出C3D深度卷积网络,学习视频的时空特征,提升深度学习方法在视频分类的效果。后来Joao等人在
C3D网络的基础上提出双流的深度卷积网络I3D~\cite{carreira2017quo},同时使用帧图像和光流作为输入,进一步提升了视频分类的准确率。深度学习模型同样成功地应用在自然语言处理上,例如Tomas等人提出词向量的分布式表达(Distributed representation)的word2vec模型~\cite{mikolov2013distributed},使用稠密的词向量表达解决了传统的词向量的one-hot编码的维数灾难问题,而且word2vec的这种表达使得词向量具有词的语义信息,可以直接用来计算不同词之间的相似度。Sepp等人~\cite{hochreiter1997long}提出长短期记忆网络(Long Short-term Memory,LSTM),后来Kyunghyun等人~\cite{cho2014learning}进一步提出门控循环单元(Gated Recurrent Unit,GRU),捕捉序列号的数据的上下文信息。而在视频和自然语言的跨模态领域同样也有大量基于深度学习的工作~\cite{venugopalan2015sequence,yao2015describing,otani2016learning,dong2018predicting,sun2019videobert,miech2019howto100m},虽然基于深度学习的技术在视频检索领域有了很大进展~\cite{mithun2018learning,miech2018learning,dong2019dual,liu2019use},但是这个领域仍然有很大的提升空间。因此本文基于深度学习研究即席视频检索,提高视频检索的效果,具有深远的应用价值和研究价值。


\section{本文研究内容}
如表格~\ref{tab:method-diffs}所示,在大型的即席视频检索挑战的基准线如TRECVID系列,早期的大部分最先进的技术都是基于概念建模,即查询和视频都以概念向量表达~\cite{le2016nii,foteini2016iti,nguyen2017vireo,ueki2017waseda}。对于查询表达,通过选择一个或几个与查询文本在词法上~\cite{lu2016event,ueki2017waseda}或者在语义上~\cite{markatopoulou2017query,snoek2017university}匹配的概念来表示查询,而对于视频表示,在图像或视频识别数据集(如ImageNet~\cite{deng2009imagenet},EventNet~\cite{ye2015eventnet}和FC-VID~\cite{jiang2018exploiting}等等)上预训练的深度卷积神经网络被用来预测视频的概念,从而组成视频的概念向量。尽管这种方法有着很好的解释性,但是基于概念的表示方法也有两个缺点:首先,选择合适数量的合适概念来可靠地描述视频内容和查询是十分困难的;第二,查询表达的某个特定的概念的重要性是根据经验进行估计的(例如根据某个概念与查询的语义相关性来估计该概念的重要性),并且这种方法在跨模态的相似度计算也不是最优的。

由于基于概念的方法有这些不足,本文研究一种不需要进行显式的概念建模、概念匹配和概念选择的完全基于深度学习的方法。在以文本检索图像领域,Faghri等人~\cite{faghri2017vse++}提出的VSE++模型使用门控循环单元(Gated Recurrent Unit, GRU)网络来对查询进行建模,而Dong等人~\cite{dong2018predicting}提出的Word2VisualVec(W2VV)模型使用多尺度的文本编码策略,被证明了这种编码策略比仅使用GRU的要好,Mithun等人~\cite{mithun2018learning}则对VSE++进行了改进以适应于视频和文本的检索任务。而基于深度学习技术的查询表达方法很少被用在大型的即席视频检索任务上,如TRECVID系列。而且在即席视频检索上的大多数方法都是基于单一的公共空间,即使使用了多个不同的特征表达也只是用特征拼接的方式进行了融合~\cite{li2018renmin,wu2019hybrid},而在视频与文本检索上,Miech等人~\cite{miech2018learning}使用了四个视频特征的表达,并且为每个特征学习一个独立的公共空间,最后使用平均融合的方法融合这四个空间。而本文在大型即席视频检索上研究为多个文本的表达学习独立的公共空间并且融合这些独立的公共空间。

本文的贡献可归纳为如下几个方面:
\begin{enumerate}[1.]
\item 技术上,通过对W2VV(原本用于图像-文本匹配)进行改进并再用在即席视频检索任务上,我们提出了一种新的方法W2VV++,有效地提高了即席视频检索的准确率。而对比之前基于概念建模的方法,W2VV++不用对查询表达进行启发式地关键词匹配。基于W2VV++模型,本文继续提出为多个文本编码器学习独立的公共子空间并融合这些子空间的方法,进一步提高了即席视频检索的效果,超过了当前最好的视频检索算法。

\item 概念上,通过赢得TRECVID 2018的即席视频检索任务,我们成功挑战了过去基于概念的方法,首次证明了深度学习技术在大型即席视频检索任务上,如TRECVID系列,是可行的。

\end{enumerate}

\section{本文结构安排}
本文基于深度学习对即席视频检索展开研究,共分为五章,具体安排如下:

第一章为绪论,主要介绍基于深度学习的即席视频检索的研究背景与意义,研究内容和主要贡献等。

第二章综述了与本文相关的研究工作,重点介绍了即席视频检索领域的权威评测TRECVID AVS在近几年的发展,也介绍了其他一些跨模态的视频检索的工作,总结他们的优缺点并提出了本文工作的解决方案。

第三章介绍本文的整体算法设计,详细介绍了各个文本编码器的原理、视频特征的提取流程、目标函数的设计以及多个子空间的融合。

第四章介绍了本文实验所用的数据集和算法的实现细节,并且系统地评测了本文算法的性能,并与当前最优的算法做了公平的比较。

第五章对本文工作进行总结,得出结论,并对未来工作进行展望,指出进一步研究的方向。


\chapter{研究综述}

\section{即席视频检索}

本文回顾了 2016 年至 2019 年四年的 TRECVID 评测中的即席视频检索系统(Ad-hoc Video Search,AVS)~\cite{awad2016trecvid,awad2017trecvid,awad2018trecvid,awad2019trecvid}任务的效果较好的算法,因为 TRECVID 评测是最有挑战的基准,吸引了该领域包括卡耐基梅隆大学、弗吉尼亚大学、香港中文大学等学术界和阿里巴巴等工业界的一些优秀学者参与评测并且提出了很多有效的算法~\cite{}。该评测的测试数据的正确答案在参与者提交评测结果前是不公开的,并且参与者不允许为任何的查询样例调整他们的结果,这样可以公平地评价参与者的模型的泛化能力。
表格\ref{tab:method-diffs}从三个关键组成部分分析这些算法的异同,即自然语言查询是如何表示的(查询表示)、未标注的视频是如何表示的(视频表示)、在什么特征空间进行跨模态匹配(公共空间)。


\begin{table} [tbp!]
    %\renewcommand{\arraystretch}{0.95}
    \caption[AVS系统回顾]{该表格回顾了2016年至2019 TRECVID的排名前三名的AVS系统}
    \label{tab:method-diffs}
    \centering
    \scalebox{0.75}{
        \begin{tabular}{| l | l | l | l | l |}
            %\toprule
            \hline
            \textbf{排名} & \textbf{系统} & \textbf{文本表示} & \textbf{视频表示} & \textbf{公共空间} \\
            \hline
            \multicolumn{5}{|l|}{2016:} \\ \hline
            1 & Le \etal\cite{le2016nii} & \specialcell{词袋向量(Bag-of-\\words~\cite{sivic2003video})} & \specialcell{通过预训练的卷积神经网络\\(VGG-16~\cite{simonyan2014very})提取的概念特\\征向量} & 文本特征空间 \\ \hline
            2 & Markatopoulou \etal\cite{foteini2016iti} & \specialcell{基于规则的概念选择得\\到的概念特征向量~\cite{markatopoulou2017query}} & \specialcell{通过预训练的卷积神经网络\\(AlexNet~\cite{krizhevsky2012imagenet},GoogLeNet~\cite{szegedy2015going}, \\ResNet~\cite{he2016deep}, VGGNet~\cite{sivic2003video})\\提取的概念特征向量} & \specialcell{1345 维的概念\\特征空间} \\ \hline
            3 & Liang \etal\cite{liang2016inf} & word2vec~\cite{mikolov2013distributed} & \specialcell{通过预训练的卷积神经网络\\(VGG-19~\cite{simonyan2014very}, C3D~\cite{tran2015learning})\\提取的视觉特征} & 视觉特征空间 \\ \hline
            \multicolumn{5}{|l|}{2017:} \\ \hline
            1 & Snoek \etal\cite{snoek2017university} & \specialcell{词袋向量(Bag-of-\\words)~\cite{sivic2003video}} & \specialcell{由VideoStory~\cite{habibian2014videostory}模型产生\\的词袋向量(Bag-of-words)} & \specialcell{文本特征空间} \\ \hline
            2 & Ueki \etal\cite{ueki2017waseda} & \specialcell{基于规则的概念选择得\\到的概念特征向量} & \specialcell{通过预训练的卷积神经网络\\(AlexNet~\cite{krizhevsky2012imagenet}, GoogLeNet~\cite{szegedy2015going})\\提取的概念特征向量} & \specialcell{5万维的概念\\特征空间} \\ \hline
            3 & Nguyen \etal\cite{nguyen2017vireo} & \specialcell{基于规则的概念选择得\\到的概念特征向量} & \specialcell{通过预训练的卷积神经网络\\(ResNet-50)提取的概念特征\\向量} & \specialcell{2774维的概念\\特征空间} \\ \hline
            \multicolumn{5}{|l|}{2018:} \\ \hline
            1 & Li \etal\cite{li2018renmin} & \specialcell{由W2VV++~\cite{li2019w2vv++}模型产\\生的稠密向量} &  \specialcell{通过预训练的卷积神经网络\\(ResNeXt-101~\cite{xie2017aggregated}, ResNet-\\152)提取的视觉特征} & \specialcell{视觉特征空间/\\学习的隐空间} \\ \hline
            2 & Huang \etal\cite{huang2018informedia} & \specialcell{+基于规则的概念选择\\得到的概念特征向量\\+通过注意力网络得到\\的稠密向量} & \specialcell{+通过预训练的卷积神经网\\络提取的概念特征向量\\+通过预训练的卷积神经网\\络提取的视觉特征} & \specialcell{+概念特征空间\\+学习的隐空间} \\ \hline
            3 & Bastan \etal\cite{bastan2018ntu} & \specialcell{由VSE++~\cite{faghri2017vse++}模型得到\\的稠密向量} & \specialcell{通过预训练的卷积神经网络\\(ResNet-152)提取的视觉特征} & \specialcell{学习的隐空间} \\ \hline
            \multicolumn{5}{|l|}{2019:} \\ \hline
            1 & Wu \etal\cite{wu2019hybrid} & \specialcell{混合文本编码策略} & \specialcell{混合视觉特征编码策略} & \specialcell{学习的隐空间} \\ \hline
            2 & Li \etal\cite{li2019renmin} & \specialcell{由W2VV++~\cite{li2019w2vv++}模型产\\生的稠密向量} & \specialcell{通过预训练的卷积神经网络\\(ResNeXt-101, ResNet-152)\\提取的视觉特征} & \specialcell{视觉特征空间/\\学习的隐空间} \\ \hline
            3 & Ueki \etal\cite{ueki2019waseda} & \specialcell{由VSE++~\cite{faghri2017vse++}模型得到\\的稠密向量} & \specialcell{通过预训练的卷积神经网络\\(ResNet-50, ResNet-101,\\ ResNet-152)提取的视觉特征} & \specialcell{学习的隐空间} \\ \hline
            %\bottomrule
        \end{tabular}
    }
\end{table}


在 TRECVID 2016 评测中,Le等人在文献~\cite{le2016nii}中提出一种基于文本查询的解决算法。他们先用预
训练的卷积神经网络(VGG-16)对视频中的帧图像提取语义概念特征,包括主要的物体、场景属性、
物体之间的关系等。在获取视频的语义概念后,视频查询任务就转化成了文本查询任务,即以查询的文本检索视频的概念,
他们使用标准的 TF-IDF 技术来计算每个概念特征的权重。给定的查询与视频的相似度由查询
文本与视频的概念特征在文本空间中计算得出。
Markatopoulou等人~\cite{foteini2016iti}提出了一种类似的算
法,即先通过预训练的卷积神经网络(AlexNet, GoogLeNet, ResNet 和 VGGNet)来提取视
频关键帧的 1000 维 ImageNet~\cite{russakovsky2015imagenet}的概念特征和 345 维 TRECVID SIN~\cite{smeaton2009high}的概念特征,把两
类特征做拼接并且将通过这四个深度卷积网络得到的特征取平均来表示视频, 每一个概念
的维度上的数值表示该概念在视频中出现的概率。在文本表示上,他们通过显式语义分析(ESA~\cite{gabrilovich2007computing})的方
式来计算查询文本与视频的 1345 个概念之间的相关性,如果这个相关性大于一个给定的阈值,则
选择这个概念来表示这个查询文本。不同于前两种算法,Liang 等人~\cite{liang2016inf}使用 webly-labeled
learning~\cite{liang2016learning}算法来对每个查询进行建模,即先通过word2vec的稠密向量进行表示,而视频则使用VGG-19和C3D提取视觉特征。
为了解决零样本的问题,他们利用查询文本从Youtube 上爬取了弱标注的视频数据来训练模型,尽管这种方法被证明有效果~\cite{kordumova2015best},但是对于一些复杂的查询来说,从网上自动爬取相关的视频还是很困难的。因此在实际的评测中,这种方法不如前两种方法有效。

在 TRECVID 2017 评测中,尽管基于概念的查询和特征表示仍然很主流,获得评测第
一名的 Snoek 等人在文献\cite{snoek2017university}中提出一种更加优雅的表示方式——VideoStory~\cite{habibian2014videostory}。对于每个没有标注的视频,他们利用深度卷积神经网络提取深度特征,并通过线性变换将该特征转换到
VideoStory 空间进行表示。这个表示进一步通过线性变换转换为词袋向量(Bag-of-words,BoW),而每个查
询也用词袋向量进行表示。因此,视频与查询间的余弦相似度可以在词袋向量空间直接进行
计算。尽管这种方法效果很好,但是也存在两个明显的不足:第一,词袋向量(BoW)
忽略了查询语句的时序信息;第二,词袋向量的有效性依赖于合适的词典选择,但这是无法
和表示学习一同进行优化的,即一旦确定了词典,则所有文本的表示都是固定的。
相反,本文提出的通过深度神经网络来表示查询既考虑了词语
的重要性,也考虑了词语间的时序信息,并且可以端到端地进行训练优化文本查询的表示。
而第二名的Ueki等人~\cite{ueki2017waseda}使用Alexnet和GoogLeNet分别在ImageNet~\cite{deng2009imagenet}、Places~\cite{zhou2014learning}、FCVID~\cite{jiang2017exploiting}、UCF101~\cite{soomro2012ucf101}等
数据集上收集了包括物体、场景、动作等一共5万维的概念向量组成了一个很大的概念库,而Nguyen等人~\cite{nguyen2017vireo}使用了类似的方案,他们使用ResNet-50提取了2774维的概念向量,评测结果排在第三名,由此可以合适的概念选择对检索的效果影响很大。

在 TRECVID 2018 评测中,许多基于深度学习的方法来表示查询的方案出现了。
Huang等人~\cite{huang2018informedia}使用了两种方法,第一种方法是基于概念的方法,即先使用预训练的深度卷积神经网络
在YFCC~\cite{thomee2016yfcc100m},ImageNet Shuffle~\cite{mettes2016the},UCF101~\cite{soomro2012ucf101},Kinetics~\cite{carreira2017quo},Places~\cite{zhou2014learning},Google Sports~\cite{karpathy2014large}等数据上提取了一共13,626维的概念向量,而文本表示则通过基于规则的概念选择得到文本的概念向量表示,最终文本查询与候选视频的相似度在概念向量空间计算得到。第二种方法是通过联合嵌入空间的方法为文本学习连续的向量表示,即先用预训练的卷积神经网络Inception-v4~\cite{szegedy2016inception}提取视频的视觉特征,然后他们叠加了两个注意力模型Dual Attention Network~\cite{nam2017dual}和Stacked Cross-Attention Network~\cite{lee2018stacked},学习一个子空间来计算跨模态的相似度。最终,他们后融合这两种方法的结果。
而Bastan等人~\cite{bastan2018ntu}使用 VSE++~\cite{faghri2017vse++}模型,即先使用预训练的ResNet-152提取视频的视觉特征,使用Gated Recurrent Unit(GRU)~\cite{cho2014learning}编码查询文本,然后为两个模态表示学习一个公共的子空间,该方法获得了评测的第三名。而本文研究的基于深度学习的方法W2VV++取得评测的第一名。

在 TRECVID 2019 评测中,排名前三的方法都是基于深度学习的。
排在第一名的Wu等人\cite{wu2019hybrid}使用了混合的序列编码策略对视频特征和文本的词向量进行聚集建模。由于查询句子和视频都是序列的模态,即视频由一系列的图像组成,句子由一系列的单词组成,因此视频深度特征通常是先对视频提取帧图像,然后再对图像使用卷积神经网络ResNet-152到的是一系列的图像特征,同样句子先经过word2vec模型得到一系列的单词的词向量。为了得到最终的视频表示和句子表示,他们使用平均池化、GRU~\cite{cho2014learning}、NetVLAD~\cite{arandjelovic2016netvlad}和图卷积网络~\cite{mao2018hierarchical}的方式对这些序列的特征进行聚集表达成视频向量和句子向量,然后通过拼接的方式相对应地融合这些特征来作视频和句子的最终向量表示,最后他们为这两个模态的向量联学习一个公共的子空间。

\section{跨模态匹配的特征融合}
即席视频检索本质上是一个以文本搜视频的跨模态匹配问题,因此本文也关注到一些基于深度学习解决文本-视频的跨模态匹配任务的工作。
因为文本-视频的跨模态匹配的核心是计算文本与视频的相关度,而文本与视频具有的异构性的特点,因此需要把它们投到异构公共的空间计算,而在这之前则需要分别对文本和视频进行向量化表示。
视频是由一序列的帧图像和一段音频组成,既包括了静态的对象也含有动态的信息,是很复杂的多媒体数据,因此对视频进行准确地向量表示也是困难的。
Ueki等人~\cite{ueki2019waseda}为了充分提取视频的图像信息,使用了ResNet-50,ResNet-101和ResNet-152三种深度卷积神经网络提取视频的帧图像视觉特征,并且为每个特征单独地学习一个文本-视频联合的公共子空间,最后使用平均后融合的方式融合这些子空间从而得到最终的视频检索结果。虽然他们使用不同的深度卷积神经网络提取视频的特征,但这几种神经网络提取的特征是很相似的,它们的网络结构是类似的,只是网络的层数逐渐更多更深,因此提取到的特征只是表达效果越来越好,但特征间缺乏互补性。考虑到视频的音频信息和动态信息,Mithun等人~\cite{mithun2018learning}除了使用ResNet-152提出视频的帧图像特征作为视频静态的物体特征,他们还使用I3D~\cite{carreira2017quo}对视频提取活动的特征和SoundNet CNN~\cite{aytar2016soundnet}来提取音频信息,然后它们使用拼接的方式融合这两种动态信息作为视频动态的活动特征,它们为物体特征和活动特征分别单独地学习公共子空间,分别在每个空间计算对应的文本与视频的相似度,最终平均这两个相似度作为文本-视频对的相似度,并根据这个相似度对候选视频排序,它们的模型结构图如图~\ref{fig:related_fig1}所示。

\begin{figure*}[tbh!]
    \centering
    \includegraphics[width=\linewidth]{figures/related_fig1}
    \caption[Mithun等人的活动-文本空间和物体-文本空间模型结构图]{\textbf{活动-文本空间和物体-文本空间模型结构图},来自~\cite{mithun2018learning}。}
    \label{fig:related_fig1}
\end{figure*}

而在视频片段检索领域,Yu等人~\cite{vsrel2020}使用了类似的做法,他们首先使用预训练的卷积神经网络VGG提取视频的帧图像特征,并使用平均池化的操作得到视频的物体特征,另外,他们还使用一个双流的卷积神经网络temporal segment network(TSN)~\cite{wang2016temporal}来提取视频的活动特征,该双流网络同时以帧图像和视频的光流作为输入。同样,他们单独地为这两类特征学习一个公共空间,并使用平均的方式后融合这两个公共空间。

考虑到视频复杂信息,Liu等人~\cite{liu2019use}尽可能地挖掘了视频信息,他们提取了视频的物体特征、场景特征、动作特征、人脸特征、字符特征(optical character recognition,OCR)、言语(speech)特征和音频(audio)特征。他们首先使用深度卷积神经网络SENet-154~\cite{}模型提取视频帧图像的物体特征,在Places365~\cite{}数据上预训练的DenseNet-161~\cite{}提取帧图像的场景特征,双流卷积神经网络I3D提取连续帧图像的动作特征。而对于人脸特征,他们先使用人脸目标检测网络SSD~\cite{}来提取帧图像中的人脸,然后再使用在VGGFace2~\cite{}数据上预训练的ResNet-50提取每个检测出的人脸的特征。对于OCR特征,他们先使用文本检测模型PixelLink~\cite{}来定位图像中的文本位置,然后每个文本框内的图像区域经过一个在文本识别数据Synth90K~\cite{}的模型~\cite{}得到一序列字符串的文本,最后文本的每个单词的向量由预训练的word2vec模型提取。对于言语特征,他们使用谷歌云的API来对声音提取单词,然后得到的句子的单词的向量同样由word2vec模型提取。对于音频特征,他们使用在Youtube-8m~\cite{}上预训练的audio CNN~\cite{}提取。最后,他们使用平均池化的方法来处理这些帧图像的物体特征、场景特征、动作特征和人脸特征,而使用NetVLAD的方法聚集OCR特征、言语特征和音频特征,从而最终得到视频级的相应的特征表示。为了融合所有的这些视频级特征,他们首先将这些特征做了线性投影,将所有特征转换到相同维度的特征,然后计算两两特征间的关系,然后根据这个关系对转换后的特征进行缩放处理得到新的特征表示,


在最近的由 Dong 等人\cite{dong2019dual}提出的对偶编码在即席视频检索上取得了很好的效果,他们
通过使用词袋模型(Bag-of-words)编码句子的全局信息,Gated Recurrent Unit(GRU)~\cite{cho2014learning}编码句
子的时序信息和卷积网络(CNN)编码句子的局部信息,但他们只是将这三部分信息进行简单
的拼接,并不能最优地利用这些信息的互补性,本文将研究通过构建多子空间的方式来充分
有效地利用这些信息。


\section{本章小结}

\chapter{算法设计}

\section{问题描述}
对于给定一个以自然句子表示的即席查询$s$,共包含$l$个单词$\{w_1,w_2,\ldots,\\w_l\}$,本文研究的目标是建立一个
视频检索系统,即从$n$个未被标注的视频集$\mathcal{V}=\{v_1,v_2,\ldots,v_n\}$中搜索出与该查询语义相关的视频。问题的关键是构建
一个跨模态的相似度函数$cms(s,v) \in \mathbb{R}$,使得相关的句子视频对$(s,v^+)$的相似度比不相关的句子视频对$(s,v^-)$
更大。相应地,根据这个相似度对视频集合$\mathcal{V}$中的所有视频降序排序,则在查询结果中相关的视频$v^+$会排在不相关的视频$v^-$的视频前,从而检索系统返回排序靠前的视频。而$s$和$v$的相似度$cms(s,v)$计算需要将$s$和$v$在公共的跨模态空间进行合适的向量化表示,在得到视频在公共空间的向量化表示前需要先用深度卷积网络来提取视频的特征,而句子则需要用文本编码器如bow得到句子的向量。设$e_t$是一个可以将句子编码成一个$d_t$维向量的文本编码器,即$e_t(s) \in \mathbb{R}^{d_t}$,如果有$k$个文本编码器$\{e_{t,1},\ldots,e_{t,k}\}$同时对查询句子进行编码,则会得到$k$个不同维度$\{d_{t,1},\ldots,d_{t,k}\}$的句子向量。然后需要对这些查询句子向量和视频特征投影到公共的跨模态空间,
设$\mathbf{s}$和$\mathbf{v}$是查询句子与视频
在公共空间的向量化表示,则跨模态的相似度由余弦相似度得到:
\begin{equation}
    \label{eq:cosine-sim}
    \begin{aligned}
        cms(s,v) := \frac{\mathbf{s}^T\mathbf{v}}{\left\| \mathbf{s} \right\| \cdot \left\| \mathbf{v} \right\|}
    \end{aligned}
\end{equation}

本文着眼于查询表示学习,即由查询$s$获得获得$k$个文本编码向量$\{e_{t,1}(s),\\ \ldots,e_{t,k}(s)\}$,并有效地利用这些文本编码器,从而得到查询$s$在公共空间的向量表示$\mathbf{s}$,而视频可以像之前的工作一样由深度卷积网络得到
的特征或者概念向量表示。

\section{句子编码器}
本文是建立在Word2VisualVec(W2VV)~\cite{dong2018predicting}模型的基础上,W2VV模型原本是用在图像描述或者视频描述的检索任务上,概念模型图如\ref{fig:base-w2vv}所示,该模型共包含两个子网络,即一个句子编码网络,把一个句子向量化和一个转换网络,将句子向量转换到视觉特征空间中。W2VV模型的句子编码网络由三个文本编码器并行对句子进行向量化,即bag-of-words (bow),word2vec (w2v)和Gated Recurrent Unit (GRU),最后使用拼接的方式融合由这三种编码器得到句子向量,而转换网络是由多层全连接层组成的多层感知机,他们的训练目标是最小化在视觉空间中的句子向量和视频向量之间的均方误差(Mean Square Error,MSE)。考虑到W2VV模型使用的GRU编码器只取了最后时刻的输出向量,并且只通过简单地拼接这三种向量,作为句子的向量表示不是一种最优的方法,训练时只考虑减小正样本视频-文本对的欧式距离,并没有增大考虑负样本视频-文本对的距离,而对于一个新来的查询句子,可能会与负样本视频的距离同样很小,因此这也不是一个最优的训练目标。
\begin{figure*}[tbh!]
    \centering
    \includegraphics[width=\linewidth]{figures/w2vv-model}
    \caption[Dong等人~\cite{dong2018predicting}提出的W2VV模型的概念图]{\textbf{W2VV模型概念图,来自~\cite{li2017multi}。该模型包含两个子网络,第一个是句子编码网络,句子同时被三种编码器进行向量化编码,即bag-of-words (bow),word2vec (w2v),和Gated Recurrent Unit (GRU),第二个是转换网络,是由两层全连接组成的感知机,将拼接的三种句子编码向量投影到视觉特征空间中,并在训练时使用正样本的句子-视频来优化平均平方误差的目标函数。}}
    \label{fig:base-w2vv}
\end{figure*}

鉴于此,本文基于W2VV模型使用一个更优的编码策略和一个更优的目标函数用于训练,即对于GRU,本文采用平均池化的方法处理其所有的隐藏状态输出表示该文本的GRU输出,并且使用improved triplet ranking loss (ITRL)~\cite{faghri2017vse++}作为训练目标,该训练目标在训练时同时增大正样本视频-文本对的相似度并且减小负样本视频-文本对的相似度,为了方便,本文将此模型命名为W2VV++,模型结构图如图~\ref{fig:w2vv++}所示。而对于不同的编码器之间的融合,本文在W2VV++的基础上继续研究多子空间融合的方法,即为不同的编码器独立构建公共的跨模态空间,最后在同一个模型内融合这些公共空间得到最终的跨模态相似度计算,本文将这种融合模型称为Sentence Encoder Assembly (SEA),模型结构图如图~\ref{fig:sea}。接下来,本文将介绍本文使用的三种文本编码器,即bag-of-words (bow),word2vec (w2v)和Gated Recurrent Unit (GRU)。

\begin{figure*}[tbh!]
    \centering
    \includegraphics[width=\linewidth]{figures/w2vv++v2}
    \caption[即席视频检索模型W2VV++概念图]{\textbf{即席视频检索模型W2VV++概念图。该模型基于W2VV模型~\cite{dong2018predicting}进行改进,使用平均池化的操作利用了GRU编码的所有时刻的隐藏状态输出。为了利用充足的负样本句子-视频对,W2VV++的训练优化目标是improved triplet ranking loss (ITRL)~\cite{faghri2017vse++},而不是W2VV模型的只考虑了正样本的句子-视频对的平均平方误差。}}
    \label{fig:w2vv++}
\end{figure*}

\begin{figure*}[tbh!]
    \centering
    \includegraphics[width=\linewidth]{figures/sea}
    \caption[即席视频检索模型SEA概念图]{\textbf{即席视频检索模型Sentence Encoder Ensembles (SEA)概念图。该模型为每个句子编码器构建独立的公共跨模态空间,即每个句子编码器输出的句子向量都经过转换网络投影到独立的公共空间中,而最终的句子与视频的相似度由句子与视频在所有公共空间中计算的相似度取平均得到,SEA采用多空间多目标函数的训练策略,即为每个公共空间独立地计算和优化目标函数ITRL。}}
    \label{fig:sea}
\end{figure*}

\textbf{bow}:对于给定的一个句子$s$,对于bow编码技术,句子$s$的向量可以由如下式子得到:
\begin{equation}
    \label{eq:bow}
    \begin{aligned}
        e_{bow}(s) := (c(w_1,s),c(w_2,s),\cdots,c(w_m,s))
    \end{aligned}
\end{equation}
其中,$c(w,s)$计算特定单词$w$在句子$s$中出现的次数,$m$表示给定词典的单词数量,本文使用的词典由在训练数据中出现不少于5次
的单词组成,并且根据NLTK去掉其中的停用词。由于给定的词典是由训练集确定的,通常超过一万维,而AVS查询句子通常是比较短,平均长度不超过10个单词,因此bow编码的向量$e_{bow}(s)$是一个非常高维但稀疏的向量。

w2v:与bow编码器不同,w2v模型~\cite{mikolov2013efficient}通过用一个很大的文本语料训练一个两层的神经网络来产生单词的稠密的语义向量。因为w2v模型的训练目标是重建句子的上下文信息,因此不需要对训练数据进行人工标注,因此可以很容易地对成千上万的单词进行编码。对于一个给定的预训练w2v模型,设$w2v(w_i)$是可以得到句子$s$的第$i$个单词的词向量,本文通过对$s$中所有单词的w2v向量进行平均池化操作得到句子的向量,具体计算如下:
\begin{equation}
    \label{eq:w2v}
    \begin{aligned}
        e_{w2v}(s) := \frac{1}{l}\sum^l_{i=1}w2v(w_i)
    \end{aligned}
\end{equation}
本文使用的是一个500维的使用3000万张Flickr图像的英文标签训练的w2v模型~\cite{dong2018predicting},共支持超过170万个常见的单词向量化。

GRU:和W2VV相似,本文同样使用Gated recurrent unit (GRU)~\cite{cho2014learning}作为序列模型对句子进行建模。GRU在特定的$t$时刻的输入
是句子中第$t$个单词的词嵌入向量,设为$e(w_t)$,该向量是从一个词嵌入矩阵$W_e$中对应的单词的词嵌入向量得到的,并作为网络的可学习参数和GRU一同进行端到端训练。而当前时刻GRU的输出,设为$h_t$是通过联合当前词嵌入向量$e(w_t)$和前一时刻GRU的输出向量$h_{t-1}$由如下式子得到:
\begin{equation}
    \label{eq:gru}
    \begin{aligned}
        & z_t = \sigma_g(W_z e(w_t) + U_z h_{t-1} + b_z), \\
        & r_t = \sigma_g(W_r e(w_t) + U_r h_{t-1} + b_r), \\
        & \widetilde{h_t} = \sigma_h(W_h e(w_t) + U_h (r_t \circ h_{t-1}) + b_h), \\
        & h_t = (1-z_t) \circ h_{t-1} + z_t \circ \widetilde{h_t},
    \end{aligned}
\end{equation}
其中$z_t$ 和$r_t$表示在$t$时刻的更新门向量和重置门向量,$W$,$U$和$b$表示门的仿射转换参数,每个门输出前带有一个特定的激活函数,其中
$\sigma_g$表示sigmoid函数,$\sigma_h$表示双曲正切函数,操作$\circ$为两个向量的哈达玛积,即逐元素相乘。

对于基于GRU的句子编码,W2VV只取最后时刻的输出向量,即$h_l$,而本文通过对所有时刻的输出向量进行平均池化操作,考虑了所有的中间时刻的输出状态,即:
\begin{equation}
    \label{eq:gru-mean}
    \begin{aligned}
        e_{gru}(s) := \frac{1}{l}\sum^l_{i=1}h_i
    \end{aligned}
\end{equation}
对于GRU词典,与bow词典类似,但是GRU词典还包含停用词,因为它们在自然语句中是含有意义的上下文信息。

W2VV++模型:如图~\ref{fig:w2vv++}所示,W2VV++模型通过使用拼接的方式融合这三种编码器输出的向量,设$e_{ms}(s)=[e_{bow}(s);e_{w2v}(s);e_{gru}(s)]$,其中$[;;]$表示向量拼接操作,向量$e_{ms}(s)$通过转换网络投影到与视频的公共跨模态空间,在该公共空间计算查询句子与视频的相似度$cms(s,v)$。

SEA模型:如图~\ref{fig:sea}所示,SEA模型通过为每个编码器构建一个独立的公共跨模态空间,在每个公共空间单独计算查询句子与视频的相似度$cms_i(s)$,$i$表示在第$i$个公共空间,并使用相加的方式融合各个空间中的相似度作为最终的查询句子与视频的相似度,即$cms_1(s,v)+cms_2(s,v)+cms_3(s,v)$。

%对于多种编码方式(bow, w2v, GRU)对句子进行编码,本文使用两种策略对这些编码方式进行融合,即:
%\begin{itemize}
%    \item W2VV++:使用与W2VV相同的方式,即向量拼接融合,即$\bm{\mathbf{ms}}(s)=[\bm{\mathbf{bow}}(s);\bm{\mathbf{w2v}}(s);\bm{\mathbf{gru}}(s)]$。本文将这种方式的模型称为W2VV++。
%
%    \item TEE:三种编码方式互相独立,分别与视频向量。本文将这种方式的模型称为text encoding ensemble(TEE)。
%\end{itemize}

\section{视频特征表示}
正如前文所言,本文研究关注于查询表示学习,因此,对于视频的表示,本文简单地使用当前最好的深度卷积网络通过过采样的方式对帧图像提取视觉特征,
并且使用平均池化的操作对帧图像进行聚合操作从而获得视频的特征表示,如图\ref{fig:video-cnn-feat}所示。

\begin{figure*}[tbh!]
    \centering
    \includegraphics[width=\linewidth]{figures/video-cnn-feat}
    \caption[视频特征提取示例图]{\textbf{视频特征提取示例图。本文使用在图像识别上预训练的深度卷积网络以过采样的方式提取视频的帧图像的特征,并用平均池化的操作聚合帧图像特征从而得到视频级特征。}}
    \label{fig:video-cnn-feat}
\end{figure*}

本文使用两个深度卷积网络模型进行视频的特征提取:
ResNeXt-101~\cite{xie2017aggregated}和ResNet-152~\cite{he2016deep}。对于每个模型,本文选择模型的分类层输出作为特征,维度为2048维。
对于给定一个视频,本文以0.5秒为间隔对视频的帧图像进行均匀采样。每张采样的图像的大小调整为$256\times256$,然后以$224\times224$大小
的窗口对该图像与其水平翻转得到的图像的四个角和中央进行裁剪,得到该图像的10张子图,这10张子图分别经过卷积网络提取特征并且进行
平均池化操作,得到该图像的特征表示。相应地,2048维的视频特征由帧图像特征进行平均池化得到。为了方便表示,本文将使用$ResNeXt$和$ResNet$
表示经过这两种卷积网络得到的视频特征,$ResNeXt$-$ResNet$表示这两种特征进行拼接得到的4096维的视频特征,设视频$v$的深度卷积特征向量为$f(v) \in \mathbb{R}^{d_v}$。

原则上,本文的模型可以使用任何的视频特征表示,包括概念特征向量~\cite{markatopoulou2017query,lu2016event,merler2012semantic}和3D卷积网络特征~\cite{mithun2018learning}。

\section{转换网络}
由前文可在,计算查询句子$s$和视频$v$的跨模态相似度$cms(s,v)$需要将$s$和$v$在公共的跨模态空间进行向量化表示,则在得到句子的编码向量表示和视频的特征表示后,关键一步是将这两个模态的向量化表示转换到维度相同向量。转换网络以上一层的编码网络输出作为输入,并将输入的向量转换成另一个维度的向量。

%和转换网络用于将之前的网络输出进行非线性仿射变换,即对于W2VV++模型,将$\bm{\mathbf{ms}}(s)$进行转换,对于TEE模型,分别对$\bm{\mathbf{bow}}(s)$,$\bm{\mathbf{w2v}}(s)$和$\bm{\mathbf{gru}}(s)$进行转换,
%把文本编码得到的向量转换到一个公共空间,得到向量$\bm{\mathbf{s}}$,使得句子与视频的相关性可以由公式\ref{eq:cosine-sim}进行计算。
本文使用$n$层全连接层实现该转换网络,设上一层网络的输出向量,即转换网络的输入向量为$g$,设$g \in \mathbb{R}^{d_1}$,则第一层全连接层
的输出向量$\bm{\mathbf{fc^1}}(s)$由作用在向量$g$的仿射变换得到,即:
\begin{equation}
    \label{eq:fc-1}
    \begin{aligned}
        \bm{\mathbf{fc}}^1(s) = \sigma(A_1 g + b_1)
    \end{aligned}
\end{equation}
其中$A_1 \in \mathbb{R}^{d_2 \times d_1}$和$b_1 \in \mathbb{R}^{d_2}$分别是全连接层的权重和偏移,$\sigma$是增强网络非线性的激活函数,本文默认的激活函数是双曲正切函数$tanh$。转换网络的剩余的全连接层计算如下公式:
\begin{equation}
    \label{eq:fc-k}
    \begin{aligned}
        \bm{\mathbf{fc}}^i(s) = \sigma(A_i\bm{\mathbf{fc}}^{i-1}(s) + b_i), i=2,...,n.
    \end{aligned}
\end{equation}

%本文的句子编码网络和转换网络是端到端地进行训练的,把网络的所有可学习的参数$\{W_z,U_z,b_z,W_r,U_r,b_r,W_h,U_h,b_h,W_e,A_1,b_1,\cdots,A_k,b_k\}$表示成$\theta$,相应地,相似度函数参数表示为$cms(s,v;\theta)$。

W2VV++模型:对于该模型,文本端的编码器的输出向量为$e_{ms}(s) \in \mathbb{R}^{d_t}$,因此文本端的转换网络的输入是$e_{ms}(s)$,设文本端的转换函数为$\bm{\mathbf{fc}}_t^n(s)$,则文本端的转换网络输出为$\mathbf{s} = \bm{\mathbf{fc}}_t^n(s)$,$\mathbf{s} \in \mathbb{R}^{d_c}$,对于视频端,设视频端的转换函数为$\bm{\mathbf{fc}}_v^n(v)$,则视频端的转换网络的输入是视频的深度卷积特征向量$f(v) \in \mathbb{R}^{d_v}$,转换网络的输出是$\mathbf{v} = \bm{\mathbf{fc}}_v^n(v)$,$\mathbf{v} \in \mathbb{R}^{d_c}$。因此在公共的跨模态空间,查询句子与视频的相似度可以由公式\ref{eq:cosine-sim}计算得到。

SEA模型:对于该模型,文本端的编码器输出向量分别为$e_{bow}(s) \in \mathbb{R}^{d_{t,1}}$,$e_{w2v}(s) \in \mathbb{R}^{d_{t,2}}$和$e_{gru}(s) \in \mathbb{R}^{d_{t,3}}$,如图~\ref{fig:sea}所示,该模型为每个文本编码器构建独立的公共跨模态空间,即每个编码器输出的向量都经过一个独立的转换网络,分别设为$\bm{\mathbf{fc}}_{t,1}^n(s)$,$\bm{\mathbf{fc}}_{t,2}^n(s)$,$\bm{\mathbf{fc}}_{t,3}^n(s)$,则文本端的转换网络输出分别为$\mathbf{s}_1 = \bm{\mathbf{fc}}_{t,1}^n(s)$,$\mathbf{s}_2 = \bm{\mathbf{fc}}_{t,2}^n(s)$,$\mathbf{s}_3 = \bm{\mathbf{fc}}_{t,3}^n(s)$,$\mathbf{s}_1 \in \mathbb{R}^{d_{c,1}}$,$\mathbf{s}_2 \in \mathbb{R}^{d_{c,2}}$,$\mathbf{s}_3 \in \mathbb{R}^{d_{c,3}}$。同样在视频端也有三个独立的转换网络,分别将视频特征$f(v)$转换到三个公共空间中,设三个转换网络分别为$\bm{\mathbf{fc}}_{v,1}^n(v)$,$\bm{\mathbf{fc}}_{v,2}^n(v)$,$\bm{\mathbf{fc}}_{v,3}^n(v)$,这三个视频端的转换网络的输入都是视频特征$f(v)$,而输出分别是为$\mathbf{s}_1 = \bm{\mathbf{fc}}_{v,1}^n(v)$,$\mathbf{v}_2 = \bm{\mathbf{fc}}_{v,2}^n(v)$,$\mathbf{v}_3 = \bm{\mathbf{fc}}_{v,3}^n(v)$,$\mathbf{v}_1 \in \mathbb{R}^{d_{c,1}}$,$\mathbf{v}_2 \in \mathbb{R}^{d_{c,2}}$,$\mathbf{v}_3 \in \mathbb{R}^{d_{c,3}}$。查询句子与视频的相似度由三个公共空间的句子-视频相似度的平均得到,即$cms(s,v) := \frac{1}{3}(cms_1(s,v)+cms_2(s,v)+cms_3(s,v))$:

\begin{equation}
    \label{eq:cosine-sim-3}
    \begin{aligned}
        cms_1(s,v)+cms_2(s,v)+cms_3(s,v) := \\
        \frac{\mathbf{s_1}^T\mathbf{v_1}}{\left\| \mathbf{s_1} \right\| \cdot \left\| \mathbf{v_1} \right\|}+\frac{\mathbf{s_2}^T\mathbf{v_2}}{\left\| \mathbf{s_2} \right\| \cdot \left\| \mathbf{v_2} \right\|}+\frac{\mathbf{s_3}^T\mathbf{v_3}}{\left\| \mathbf{s_3} \right\| \cdot \left\| \mathbf{v_3} \right\|}
    \end{aligned}
\end{equation}

\section{目标函数}
与W2VV模型的训练目标是最小化平均欧平方误差不同,本文使用改进的三元组排序损失函数(improved triplet ranking loss,ITRL),这个目标函数原本在Faghri等人在文献~\cite{faghri2017vse++}提出用在图像-文本的相互检索上,但是也被在基于文本的视频检索的一些工作上被证明了是有效的~\cite{mithun2018learning,dong2019dual,liu2019use,wu2019hybrid,li2019w2vv++}。经典的三元组排序损失函数从训练样本中选择一个随机选择一个样本作为参照点(Anchor),然后再随机选择与参照点匹配的样本作为正样本点(Positive),而再随机选择与参照点不匹配的样本作为负样本点(Negative),从而构建出一对三元组样本对(Anchor,Positive,Negative)。

\begin{figure*}[tbh!]
    \centering
    \includegraphics[width=\linewidth]{figures/triplet-ranking-loss}
    \caption[三元组排序损失函数训练示例图]{\textbf{三元组排序损失函数训练示例图。三元组排序损失函数最小化参照点(Anchor)与正样本点(Positive)的距离,最大化参照点(Anchor)与负样本点(Negative)的距离。来自~\cite{schroff2015facenet}。}}
    \label{fig:triplet-ranking-loss}
\end{figure*}

如图~\ref{fig:triplet-ranking-loss}所示,在训练过程中,三元组排序损失函数在最小化参照点与正样本点的距离的同时,最大化参照点与负样本点的距离,从而参照点更能区分正样本点和负样本点。与经典的三元组排序损失函数从训练样本中随机地选择负样本构建三元组不同,在训练的每个批次(batch),ITRL只选择最难区分的负样本(即与参照点距离最近的负样本)作为参照点的负样本点,因为最难的负样本会提供更多的具有区分性的信息,因此也更能区分距离较远的负样本,训练后的模型的性能也就更好。

对于本文研究即席视频检索问题,我们的目标是对于用户输入的查询句子,视频集的视频更加具有区分性,即与用户查询越相关的视频与查询的距离更近,相反,越不相关的视频与查询句子的距离更远。因此ITRL的参照点是查询句子,而正样本点是相关的视频而负样本点是不相关的视频。
对于一个给定的训练句子$s$,对应着一个与该句子相关的正样本视频$v^+$,和与该样本不相关的一系列视频$v^-$,因此ITRL的表达式如下公式~\ref{eq:itrl}所示:

\begin{equation}
    \label{eq:itrl}
    \left\{
        \begin{aligned}
            & v^{-*} & = & \mathop{\arg\max}_{v^- \in batch}(cms(s, v^-) - cms(s, v^+)), & \\
            & ITRL(s) & = & \max(0, \alpha + cms(s, v^{-*}) - cms(s, v^+)) &
        \end{aligned}
    \right.
\end{equation}
其中$\alpha$是超参数。

W2VV++模型:该模型将查询句子$s$和视频$v$都投影到一个公共的跨模态空间,并在该公共空间计算句子与视频的余弦相似度$cms(s,v)$,并根据该相似度在训练的每个批次如公式~\ref{eq:itrl}计算并优化损失函数$ITRL(s)$,即:
\begin{equation}
    \label{eq:loss-w2vv++}
    \begin{aligned}
        loss(s;\theta) = ITRL(s)
    \end{aligned}
\end{equation}
其中$\theta$为模型中的可学习参数。

SEA模型:该模型为每个句子编码器(bow,w2v,GRU)对查询句子$s$的编码向量构建一个独立的公共跨模态空间,并将视频$v$也分别投影到这些公共空间中,在每个公共空间中可以计算句子$s$与视频$v$的相似度,每对句子-视频对工产生3个相似度,即$cms_1(s,v)$,$cms_2(s,v)$和$csm_3(s,v)$,而该句子-视频对的相似度由这三个相似度的平均得到,即$cms(s,v):=\frac{1}{3}(cms_1(s,v)+cms_2(s,v)+cms_3(s,v))$。本文认为使用最终的句子-视频相似度来计算并优化ITRL函数对于多空间学习不是一种最优的做法,因为在一个训练批次里,根据最终的句子-视频相似度来选择最难负样本对于每个跨模态公共空间不是最有效的,而应该根据句子和视频在每个公共空间计算的相似度来选择句子$s$的最难负样本视频$v^{-*}$。因此,本文选择在每个公共空间计算$ITRL_i(s), i=1,2,3$,并且最终优化三个公共空间的联合目标函数,即:
\begin{equation}
    \label{eq:loss-sea}
    \begin{aligned}
        loss(s;\theta) = \sum_{i=1}^3 ITRL_i(s)
    \end{aligned}
\end{equation}
其中$\theta$为模型中的可学习参数。与平均融合每个空间的相似度类似,本文以相等的权重融合每个空间中计算的目标函数。如图\ref{fig:negative-examples}的所示,平均融合每个空间的目标函数在训练过程中产生更多样化的难样本,这样更加有利于模型的训练。

\begin{figure*}[tbh!]
    \centering
    \includegraphics[width=\linewidth]{figures/single_vs_combined_loss}
    \caption[训练过程中对于特定的句子SEA模型自动选择的最难负样本视频的示例图]{\bfseries{训练过程中对于特定的句子SEA模型自动选择的最难负样本视频的示例图。第一列(a)根据先融合多空间的相似度再计算ITRL目标函数选择的最难负样本,而另外三列是根据为每个公共空间计算ITRL目标函数,而(b),(c),(d)分别为句子编码器$e_{bow}$,$e_{w2v}$和$e_{gru}$所投影的公共空间中选择的最难负样本。可见融合多空间的目标函数可以在每个批次产生更多样性的最难负样本。}}
    \label{fig:negative-examples}
\end{figure*}

\section{本章小结}
本章从句子编码网络、视频特征提取、转换网络和目标函数四个方面详细地介绍了本文研究的两个基于深度学习的即席视频检索算法W2VV++和SEA。W2VV++基于一个原本用于图像-句子相互检索的模型W2VV~\cite{dong2018predicting}进行了句子编码的改进,并在训练时使用更加有效的improved triplet ranking loss(ITRL)作为训练目标监督模型的训练,从而更好地适应于基于文本的视频检索任务。SEA模型在W2VV++的基础上采用多空间多目标函数的训练策略,即为每个句子编码器构建独立的公共跨模态空间,并且每个空间独立地使用ITRL作为目标函数监督各自空间的训练,最后句子与视频的相似度由句子与视频在所有公共空间计算的相似度取平均得到。


\chapter{实验}
\section{算法实现}
本研究使用PyTorch~\cite{}的深度学习框架实现W2VV++算法模型和TEE算法模型,使用RMSProp~\cite{}优化器进行模型的训练,优化器的学习率
根据实验经验设为0.0001,而其他参数使用默认值。为了防止训练时出现梯度爆炸,本研究把训练时的梯度降低l2范数倍。学习率在每次训练结束
后降为原来的0.99倍,如果模型在验证集上的性能连续三次没有提高则学习率降为原来的0.5倍。如果在验证集的性能连续10次没有提高,则模型训练
停止。本研究选择在验证集上性能最好的模型。

每次模型训练批次的大小为128对相关的句子-视频对,在每个批次给定的一个句子,本研究简单把该批次里的其余剩余视频当作该句子的不相关视频,
即该句子与这些不相关视频构成负样本,而本研究使用的最难负例来计算损失的策略,即选择这些负例中的最不相关的句子视频对来最后计算损失,
具体是余弦相似度最大的句子视频对作为最难负例。而公式\ref{eq:loss}损失函数的超参数$\alpha$设为0.2。为了避免出现过拟合,本研究在转换网络的全连接层使用概率为0.2的随机失活技术。

\section{实验数据}
为了验证本研究的算法的有效性,本研究设置了两组实验,一组是在TRECVID AVS任务上的,因为本研究的本质也是文本检索视频的任务,因此在
具有权威的视频检索数据集msrvtt10k上的实验验证本研究模型的有效性。

\subsection{TRECVID AVS实验数据}
本研究使用的训练、验证与测试集的一些基本统计如表格\ref{tab:dataset1}所示。

\textbf{训练集}:本研究使用MSR-VTT~\cite{}和TGIF~\cite{}两个数据结合作为训练集。MSR-VTT数据包含1万个网络视频片段和20万个描述视频片段内容的句子,即每个视频片段有20个描述句子。
本研究对每段视频进行均匀采样,共生成305,462帧图像。而TGIF数据包含超过10万张动图和12万个描述动图内容的句子。本研究同样对每个动图进行均匀采样,生成1,045,268帧图像。

\textbf{验证集}:本研究使用TRECVID 2016 Video-to-Text任务~\cite{}的训练集作为验证集,命名为TV16-VTT-train,用于模型训练阶段的
最优模型选择。这个集合共包含200个视频,每个视频含有2个描述视频内容的句子。对于每个视频,本文选择第一个句子作为该视频的文本查询,
而剩余的199个视频则是该查询的不相关视频。相应地,本研究使用平均倒数排名(mean reciprocal rank, MRR)作为评价模型在该验证集性能的指标。
\begin{table} [tbh!]
    \caption[AVS数据集]{这是AVS数据集的基本统计描述}
    \label{tab:avs-dataset}
    \centering
    \scalebox{0.96}{
        \begin{tabular}{@{}l r r r r r@{}}
            \toprule
            \textbf{数据集名称} & \textbf{镜头数} & \textbf{帧数} & \textbf{句子数} & \textbf{bow词典大小} & \textbf{GRU词典大小} \\
            \hline
            \emph{训练集:} & & & & & \\
            MSR-VTT & 10,000 & 305,462 & 200,000 & \multirow{2}{*}{11,147} & \multirow{2}{*}{11,282} \\
            TGIF & 100,855 & 1,045,268 & 124,534 & & \\
            \hline
            \emph{验证集:} & & & & & \\ 
            TV16-VTT-train & 200 & 5,941 & 200 & - & - \\
            \hline
            \emph{测试集:} & & & & & \\
            IACC.3 & 335,944 & 3,845,221 & - & - & - \\
            V3C1 & 1,082,649 & 7,839,450 & - & - & - \\
            \bottomrule
        \end{tabular}
    }
\end{table}


\textbf{视频测试集}:本研究使用TRECVID AVS任务的官方测试集IACC.3(2016-2018)~\cite{}和V3C1(2019)~\cite{}。IACC.3数据集包含4,953个网络视频(600小时),视频的时长从6.5分钟到9.5分钟,平均时长为7.8分钟。
官方已经对视频做了自动镜头边界检测,共生成335,944个视频片段。本研究同样对每个视频片段进行均匀采样,共生成3,845,221帧图像。

\textbf{查询测试集}:TRECVID AVS官方每年定义30个查询来进行测试,每个查询是自然语言的句子形式,具有不同的长度和不同的语义难度。
例如“Find shots of palm trees”,“Find shots of a man with beard and wearing white robe speaking and gesturing to camera”和
“Find shots of a truck standing still while a person is walking beside or in front of it”。所有的查询均以“Find shots of”的短语开头,因此在测试时可以很容易地去掉。

\textbf{性能指标}:本研究使用TRECVID AVS任务官方的评测指标,推断平均准确率(inferred average precision, infAP)~\cite{}作为该节实验的模型评价指标,而模型的总体性能是所有测试查询的infAP分数的平均值,该值越高,模型的视频检索效果越好。







\section{实验结果}
\subsection{插图表格}
\begin{figure}[htbp]
\centering\includegraphics[width=5cm,height=1.32cm]{figures/logo3.pdf}
\caption[中英校名]{中英校名}
\end{figure}
\begin{table}[htbp]
\noindent\begin{minipage}{0.5\textwidth}
\centering
\caption{并排子表格}
\label{tab:parallel1}
\begin{tabular}{p{2cm}p{2cm}}
\toprule[1.5pt]
姓名 & 性别 \\\midrule[1pt]
李狗蛋 & 女 \\\bottomrule[1.5pt]
\end{tabular}
\end{minipage}
\begin{minipage}{0.5\textwidth}
\centering
\caption{并排子表格}
\label{tab:parallel2}
\begin{tabular}{p{2cm}p{2cm}}
\toprule[1.5pt]
姓名 & 性别 \\\midrule[1pt]
张狗蛋 & 女 \\\bottomrule[1.5pt]
\end{tabular}
\end{minipage}
\end{table}
\begin{table}[htbp]
\centering
\caption{并排子表格}
\label{tab:subtable}
\subtable[第一个子表格]
{
\begin{tabular}{p{2cm}p{2cm}}
\toprule[1.5pt]
姓名 & 性别 \\\midrule[1pt]
田狗蛋 & 男 \\\bottomrule[1.5pt]
\end{tabular}
}
\hskip2cm
\subtable[第二个子表格]
{
\begin{tabular}{p{2cm}p{2cm}}
\toprule[1.5pt]
姓名 & 性别 \\\midrule[1pt]
李狗蛋 & 女 \\\bottomrule[1.5pt]
\end{tabular}
}
\end{table}

\subsection{数学环境}
下面是几个数学公式的例子:\par
\begin{equation}
\begin{aligned}
P\{S_n \leq t\} &= \int_{-\infty}^{+\infty}f_{S_n}dt \notag \\
                       &= \int_0^t\frac{\lambda(\lambda u)^{n-1}}{(n-1)!}e^{-\lambda u}du \\
                       &\xlongequal{令 \lambda u=x} \frac{1}{(n-1)!}\int_0^{\lambda t}x^{n-1}e^{-x}dx\\
                       &=\frac{-1}{(n-1)!}(e^{-x}x^{n-1}{\Big|}_0^{\lambda t}-\int_0^{\lambda t}e^{-x}dx^{n-1})\\
                       &=\frac{-1}{(n-1)!}e^{-x}x^{n-1}{\Big|}_0^{\lambda t}+\frac{1}{(n-2)!}\int_0^{\lambda t}e^{-x}x^{n-2}dx
\end{aligned}
\end{equation}\par
再来几个:
\begin{equation}
\begin{aligned}
\lambda &=\left (1+\frac{\left(\frac{\bar{X}-\bar{Y}}{\sqrt{((\frac{1}{n}+\frac{1}{m})\sigma^2)}}\right)^2}{\left(\sqrt{\frac{\sum\limits_{i=1}^n(X_i-\bar{X})^2+\sum\limits_{i=1}^m(Y_i-\bar{Y})^2}{(m+n)\sigma^2}}\right)^2(m+n-2)}\right)^{\frac{n+m}{2}} \\ \notag
            &=\left(1+\frac{T^2}{n+m-2}\right)^{\frac{n+m}{2}}\\
 其中\quad T^2 &=\left(\frac{\frac{\bar{X}-\bar{Y}}{\sqrt{((\frac{1}{n}+\frac{1}{m})\sigma^2)}}}{{\sqrt{\frac{\sum\limits_{i=1}^n(X_i-\bar{X})^2+\sum\limits_{i=1}^m(Y_i-\bar{Y})^2}{(m+n)\sigma^2}}}}\right)^2
\end{aligned}
\end{equation}
%要插入本科签名的最后一个章节,插入命令使用\input{}

%本科签名
%\autograph


%参考文献
%\bibliographystyle{ref/rucbib}
\bibliographystyle{unsrt}
\bibliography{ref/reference}
%\nocite{*}
\addcontentsline{toc}{chapter}{参考文献}

%附录
\appendix
\chapter{即席视频检索结果展示} \label{app:app-a}
\begin{figure*}[tbh!]
    \centering
    \includegraphics[width=\linewidth]{figures/search_result_1}
    \caption[SEA模型在V3C1上的即席视频检索结果展示]{\textbf{SEA模型在V3C1上的即席视频检索结果展示。这里的查询句子均来自TRECVID AVS评测,展示了检索结果的前5个视频。}}
    \label{fig:search_result_1}
\end{figure*}

\begin{figure*}[tbh!]
    \centering
    \includegraphics[width=\linewidth]{figures/search_result_2}
    \caption[SEA模型在V3C1上的即席视频检索结果展示(续)]{\textbf{SEA模型在V3C1上的即席视频检索结果展示(续)。这里的查询句子均来自TRECVID AVS评测,展示了检索结果的前5个视频。}}
    \label{fig:search_result_2}
\end{figure*}



%致谢
\begin{acknowledge}%致谢
    光阴荏苒,日月如梭,三年的研究生生涯即将画上句号。在这三年专研学术、追求真理的旅途中,虽然我历经坎坷与艰辛,在科研任务停滞不前的时候都会陷入迷茫、怀疑人生,但正是这样曲折的旅途让我重新认识了自己,学会了很多,不仅包括在学术上的研究能力,更增长了在困难面前迎难而上的勇气与信心,也成长了很多,懂得更好地管理自己未来的人生,为成为国民表率、社会栋梁继续努力奋斗。在这里,我要感谢在这段旅途中陪伴和帮助过我的良师益友。

    首先我要感谢我的导师李锡荣老师,他是我在专研学术之路的引路人,也是我人生道路的灯塔。李老师治学严谨,刻苦专研,高瞻远瞩,对我在研究生期间的研究选题、研究方法提供了很多的指导和建议,李老师对学术研究的热情和态度以及实事求是的精神对我产生了深远的积极影响。李老师不仅在我研究上遇到困惑时对我耐心教导,教会我科学地思考问题,而且李老师言传身教,帮我修改代码,教会我搭建展示系统,纠正我的代码风格,帮我修改论文,也教会我很多论文写作的技巧。我也很感谢李老师在我生病住院时给予了很大的帮助和关心,在我遇到挫折时不断鼓励和支持我,帮我度过了一个又一个难关。
    在此我再次向李老师表示衷心的感谢!

    另外,我要感谢中国人民大学人工智能与媒体计算实验室的师兄师姐和小伙伴们,与你们相处的日子非常愉快,也很怀念与你们一起打比赛、一起讨论、一起出去聚餐的时光,是你们让我那原本枯燥的研究生生活变得十分丰富多彩。我还要感谢我的室友们对我的包容,感谢你们在我生病时给予关心与帮助。
\end{acknowledge}




\end{document}  






